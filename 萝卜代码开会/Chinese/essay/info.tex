\PaperTitle{粒子群优化算法综述} % Article title


\Authors{王汝芸\textsuperscript{1}*} %韩梅梅\textsuperscript{2}} % Authors
\affiliation{
	\quad
	\textsuperscript{1}\textit{山东师范大学信息科学与工程学院}
	\qquad
	%\textsuperscript{2}\textit{北京大学物理学院理论研究所}
	\qquad
	*\textbf{通讯作者}: ruyunw@163.com
} % Author affiliation

\Abstract{
	\phantom{田田}
	本文就粒子群算法的原理、流程、社会行为分析及算法构成要素作出了详细说明,对粒子群算法的发展、研究内容、特点应用及其改进算法作出了梳理。粒子群算法是一种模拟自然界的生物活动以及群体智能的随机搜索算法。群体中的每个粒子位置代表搜索空间中的一个候选解。粒子个体之间通过位置信息的交流和学习来调整各自搜索方向。粒子群优化由于其算法简单,易于实现,无需梯度信息,参数少等特点在连续优化问题和离散优化问题中都表现出良好的效果,特别是因为其天然实数编码特点适合于处理实优化问题,近年来成为国际上智能优化领域研究的热门。作为一种重要的优化工具,粒子群优化算法已经成功地用于目标函数优化,神经网络训练,模糊控制系统,模式识别,信号处理,图形图像处理,统计学习模型参数优化,机器学习模型参数优化等领域。
	此外,本文还介绍了几个主要的算法改进方向,包括基于参数调整的改进方法、带有收缩因子的改进方法、针对离散优化问题的两个典型版本粒子群优化算法、基于遗传思想和梯度信息的改进策略,及算法在约束优化和多目标优化两类复杂环境中的解决方案。
	最后,通过实验,探究了影响粒子群优化算法的因素。通过参数的改变,探究了迭代次数和种群规模对算法收敛精度的影响。通过8个经典测试函数的仿真,将标准粒子群算法及其各个改进算进行比较,统计不同算法在不同函数中的表现,得出粒子群算法在参数改进、算法混合、学习算子改进及拓扑结构改进等方面各自的特点,对日后的算法改进起到一定的指导作用。
	
}


\Keywords{\phantom{田田}粒子群优化算法\quad计算智能\quad演化算法} % 如不需要关键词可直接删去花括号中内容

